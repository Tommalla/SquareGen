\documentclass[a4paper,10pt]{article}
\usepackage{polski}
\usepackage[utf8]{inputenc}
\usepackage{graphicx}
\usepackage{listings}
\usepackage{xcolor}
\usepackage{tabularx}
\usepackage{adjustbox}
\usepackage{amssymb}
\usepackage{amsmath}
\usepackage{float}

%opening
\title{Tworzenie słów o maksymalnej liczbie kwadratów - analiza metod opartych o przeszukiwanie Monte-Carlo}
\author{Tomasz Zakrzewski, tz336079}

\begin{document}

\maketitle

\section{Wstęp}
Ten dokument przedstawia analizę jakości i porównanie 3 algorytmów, przy tworzeniu słów zadanej długości ($n$; $200$ oraz $500$) 
nad alfabetem binarnym ($\{0, 1\}$), zawierających największą możliwą liczbę kwadratów. Algorytmy zastosowane w próbach, to: NMCS
- Nested Monte-Carlo Search[1], NRPA - Nested Rollout Policy Adaptation[2], BeamNRPA - Beam Nested Rollout Policy Adaptation[3].
W tej pracy opiszę po kolei eksperymenty przeprowadzane na każdym z tych algorytmów, zachowując kolejnośc w jakiej je implementowałem,
włącznie z kolejnością wzbogacania standardowych pomysłów o swoje własne.

\section{NMCS}
Algorytm w standardowej implementacji dawał się bez większych problemów wykonywać dla $level \in \{0, 1\}$. Dla wyższych wartości, robił
się wystarczająco wolny, aby kończyła mi się cierpliwość. Tutaj też pojawił się pierwszy pomysł optymalizacyjny - zapamiętywanie najlepszego
wyniku między wykonaniami algorytmu. W efekcie, lepiej eksplorujemy lokalne maksima, ale istnieje ryzyko ugrzęźnięcia w takim maksimum. 
W momencie implementowania i testowania NMCSa, problem ten pominąłem uruchamiając po prostu program ponownie jeśli zbyt długo zwracał ten 
sam wynik. Przy kolejnych algorytmach podam pewne sposoby radzenia sobie z tym.

Poniżej prezentuję tabelę typowych wyników otrzymywanych NMCSem dla danych $n$ i $level$ po pewnej ilości iteracji: \\
\begin{adjustbox}{center}
\begin{tabularx}{0.7\linewidth}{|X|X|X|X| }
  \hline
  & \multicolumn{3}{|X|}{$level$} \\
  \hline
  $n$ & $0$ & $1$ & $2$ \\
  \hline
  $200$ & $50 - 85$ & $80 - 155$ & $140-$brak cierpliwości \\
  \hline
  $500$ & $70 - 120$ & $159 - 260$ & brak cierpliwości \\
  \hline
\end{tabularx}
\end{adjustbox}

Widać wyraźną tendencję do poprawy z rosnącym $level$, ale też duży koszt obliczeniowy tej metody.

\section{NRPA}
Kolejnym zaimplementowanym algorytmem był NRPA. Ten również powstał w dosyć standardowej implementacji i szybko zaczął osiągać wyniki
lepsze niż poprzednik, co przypisuję głównie gradient ascentowi, wykorzystywanemu w tej metodzie zamiast prostego podmieniania wyniku na
najlepszy. W ten sposób algorytm ma szansę lepiej wyeksplorować przestrzeń rozwiązań nie grzęznąc od razu w lokalnych maksimach. Przez większość
testów używane były parametry sugerowane przez oryginalnych autorów tej metody: $\alpha = 1.0, N = 100$, gdzie $N$ oznacza ilość wywołań poziomu 
$level - 1$ na poziomie $level$. W praktyce nie zauważyłem specjalnie różnic między $\alpha$ równym $1.0$, a jakimiś mniejszymi, ale też pod tym
kątem dużo nie testowałem. Poniżej przedstawiam tabelę wyników przy tych właśnie ustalonych parametrach.

\begin{adjustbox}{center}
\begin{tabularx}{0.7\linewidth}{|X|X|X|X| }
  \hline
  & \multicolumn{3}{|X|}{$level$} \\
  \hline
  $n$ & $0$ & $1$ & $2$ \\
  \hline
  $200$ & $100$ & $100$ & $161-169$ \\
  \hline
  $500$ & $250$ & $250$ & $290-444$ \\
  \hline
\end{tabularx}
\end{adjustbox}

Warto zauważyć, że NRPA z odpowiednio dobranymi parametrami jest dla danego $level$ wyraźnie szybszy niż NMCS, a mimo to generuje lepsze wyniki.
W dodatku, algorytm ten jest znacznie stabilniejszy w jakości produkowanych ciągów, nigdy nie spadając poniżej pewnych ``porządnych'' wyników.

W tym momencie implementacji i testów dodałem do wszystkich algorytmów możliwość resetu pamięci oraz mutacji. Reset działa tak, jakbyśmy
uruchomili program od nowa, mutacja zaś, z danym prawdopodobieństem ($20\%$, możliwe że trochę zbyt dużo) zmienia jakoś nalepsze zapamiętane
rozwiązanie. W przypadku NMCS po prostu neguje dany znak, w przypadku NRPA i pochodnych, dla każdego przejścia w policy, z ustalonym 
prawdopodobieństem dodaje wartość $1.0$. Okazało się to wygodne do uruchamiania algorytmów na dłuższy okres czasu, ponieważ nie muszę pilnować
czy nie utknęły one w lokalnych maksimach, bo automatycznie będą z nich ``wyciągane''.

\section{BeamNRPA}
Jako udoskonalenie NRPA zaimplementowałem wersję z ławą. Zgodnie z oryginalną pracą na ten temat [3], ławę uruchamiałem tylko dla $level=1$,
dla innych poziomów działał standardowy NRPA. Próbowałem różnych rozmiarów ław, w zależności od $n$ oraz $N$. Najlepsze efekty uzyskiwałem
dla dużych $N$ i małych ław, czyli tak na prawdę nie ignorując odnóg bliskich najlepszym rozwiązaniom NRPA. Osiągane wyniki dla danych $n$ w
zależności od $level$ i rozmiaru ławy $B$. W tabelce nie uwzględniam levelu 0, ponieważ jest on równoważny wywołaniu NRPA z takim samym levelem.

Wszystkie niewymienione parametry tak jak w NRPA.

$n = 200$

\begin{adjustbox}{center}
\begin{tabularx}{0.7\linewidth}{|X|X|X| }
  \hline
  & \multicolumn{2}{|X|}{$level$} \\
  \hline
  $B$ & $1$ & $2$ \\
  \hline
  $2$ & $100-140$ & $164-169$ \\
  \hline
  $4$ & $100-145$ & $166-169$ \\
  \hline
  $8$ & $100-155$ & $167-169$ \\
  \hline
\end{tabularx}
\end{adjustbox}

\pagebreak
$n = 500$

\begin{adjustbox}{center}
\begin{tabularx}{0.7\linewidth}{|X|X|X| }
  \hline
  & \multicolumn{2}{|X|}{$level$} \\
  \hline
  $B$ & $1$ & $2$ \\
  \hline
  $2$ & $250-284$ & $346 - 423$\\
  \hline
  $4$ & $250-281$ & $321 - 364$  \\
  \hline
  $8$ & $250-268$ & $358$ - brak cierpliwości \\
  \hline
\end{tabularx}
\end{adjustbox}

\section{Wnioski, komentarze}
Spostrzeżenie, które nasuwa się szybko po przejrzeniu wyników jest takie, że kolejne metody, wraz z rosnącym stopniem skomplikowania, dosyć
stabilnie produkują lepsze minimalne wyniki. Naturalnie też łatwiej osiągają maksymalne wyniki, których nie udawało się osiągać słabszym wersjom.

Widać, że jest sporo miejsc, w których jeszcze możnaby pobawić się parametrami. Ze względu na wnioski w [2], nie ruszałem w ogóle parametru 
$\alpha$, domyślnie ustalonego na $1.0$. Możliwe że jakieś zmiany tam zaskutkowałyby lepszymi wynikami. Prawdopodobnie możnaby też wydajnościowo
poprawić implementacje (i np. zrównoleglić niektóre fazy) - możnaby wtedy uruchomić przeszukiwanie z większą głębokością, które prawdopodobnie
znalazłoby jakieś lepsze rozwiązania.

Zastanawiające są wyniki dla NRPA przy $level=1$ (osiąganie $100$ lub $250$ cały czas). Możliwe, że w mojej implementacji kryje się gdzieś jakiś
błąd, ponieważ trochę ciężko mi uwierzyć, że 2-poziomowy algorytm sprytniejszy niż NMCS nie umie wykryć lepszych rozwiązań. Niemniej nie udało mi
się do tej pory możliwego błędu zlokalizować.

\section{Odnośniki}
[1] Nested Monte-Carlo Search.
Tristan Cazenave. IJCAI, page 456-461. (2009) \\[0pt]
[2] Nested Rollout Policy Adaptation for Monte Carlo Tree Search,
Christopher D. Rosin. IJCAI, page 649-654. IJCAI/AAAI, (2011) \\[0pt]
[3] "Beam Nested Rollout Policy Adaptation", Tristan Cazenave,
Fabien Teytaud. Computer Games Workshop, ECAI 2012, pp. 1-12, Montpellier, August 2012

\end{document}